\documentclass[12pt]{article}
\usepackage[letterpaper, margin=0.75in]{geometry}
\usepackage{blindtext}
\usepackage[utf8]{inputenc}
\usepackage{graphicx} 
\graphicspath{ {./} }
\usepackage{float} 
\usepackage{karnaugh-map}
\usepackage{rotating}
\usepackage{pdflscape}
\usepackage{caption}
\usepackage{booktabs}
\usepackage{subcaption}
\usepackage{placeins}

\newcommand{\foo}{\hspace{-2.3pt}$\bullet$ \hspace{5pt}}

\begin{document}
\title{\textbf{Tesseract Project Revisit:\\Specification and Requirements\\ MSE 3302 B}}
\author{
  Alan Harris\\
  250901911
  \and
  Robert Potra\\
  250914807
  \and
  Andrew Randell\\
  250911270
  \and
  Kevin Wang\\
  250908180
}
\date{\today}
\maketitle

\tableofcontents
\thispagestyle{empty}

\pagebreak
\setcounter{page}{1}
 
\section{Design Specifications and Requirements}
\subsection{Performance and Constraints}
\subsubsection{Required Speed}
The speed at which the system operates at must be maximized in order to reduce completion time while remaining stable with minimal overshoot and high accuracy. This speed must not exceed any limits imposed by the magnetic sensor, chassis, drive train, cube transportation system or any other subsystems. \\

The selected sensors and actuators will impose quantitative restraints on the optimal operating speed.

\subsubsection{Required Acceleration}
To provide smooth operation, acceleration will be capped at $x m/s^2$. This will ensure minimal wear on the drive train components and will reduce the probability of dropping the cube.
\subsubsection{Required Accuracy and Resolution}
The sensors which track the wall must have sufficient accuracy to maintain a constant distance. The magnetic cube sensor must have sufficient accuracy and range to detect when the cube is within grabbing range.
\subsubsection{Allowable Overshoot}
Overshoot must be kept to a minimum to prevent system crashes. The ideal response of the system is critically-damped which will yield the fastest response with zero overshoot.
\subsubsection{Power Rating and Voltage Rating for Power Supply}
The Power Supply must provide sufficient power to drive the motors at the required constant speed. The Voltage must not dip or rise outside of a specified range during operation to reduce negative disturbances on the sensors and control system.
\subsubsection{Overall Dimensions}
Must be agile enough to navigate the course and sweep the entire area. \textit{Constraints given in MSE2202B: 45 x 45 x 25 (cm)} A smaller chassis will yield more agility and allow the system to fully sweep the entire survey area. Any cubes located in the corners of the survey area present an agility challenge for the system.
\subsubsection{Operating Forces and Torques}
Operating Forces and Torques must not exceed the specified maximums of the materials used. The motors in the wheels must provide enough torque to be able to climb over the obstacles.
\subsubsection{Mass and Inertia of Components in Motion}
The weight of sensors and actuators will add to the weight of the entire robot, and must be accounted for in calculations. They will also add additional inertia, possibly reducing the accuracy of stopping and starting at set distances.
\subsubsection{Structural Frame Rigidity}
Frame must support all components used in the system. The frame must be sufficiently rigid to not deform under the weight of the sensors; actuators; controllers; power supplies; and cargo, such that it continues to operate as desired.
\subsubsection{Expected Temperature Range}
The system must comply with the Industrial Temperature standard of $-40$ to $85$ degrees Celsius. This will ensure that all components used in the system will work in a large range of environments. \\

Any heat generated by the system must be negligible from an environmental perspective, for example, the system will not emit excessive amounts of heat which can negatively affect the environment. 
\subsubsection{Expected Cleanliness}
The system must not contaminate the environment. Possible Contaminants include Oil; Grease; and Battery Acid. To comply with this regulation, sealed bearings and transmissions will be used in the drive train subsystem and dry-cell batteries will be used for power storage, both of these measures will lower the probability of environmental contamination
\subsubsection{Safety Features}
Industrial Safety standards surrounding heavy equipment must be complied with to ensure human health and safety. The system will employ measures to protect human health sand safety such a restraints and guards around moving parts; adequate shielding around components which emit large amounts of electromagnetic interference; and substantial insulation around wiring. \\

Ethics must be followed with regards to autonomous decision making. 

\clearpage
\section{Milestones and Timeline}
\subsection{Project Timeline}


\textit{Week of:}

\scalebox{1}{
\begin{tabular}{l | l}

\textbf{Jan. 14} & \textbf{Specifications and Requirement Deliverable}\\
 & $\bullet$ Create Timeline\\
 & $\bullet$ Brainstorm questions for professor\\
 & $\bullet$ Meet with group and finalize deliverable\\
\textbf{Jan. 21} & \textbf{\textit{Research Applicable Senors}}\\
 & $\bullet$ Find reliable websites for sensor specifications\\
 & $\bullet$ Decide what types of sensors we need for the robot\\
\textbf{Jan. 28} & \textbf{\textit{Do Design?}}\\
 & $\bullet$ point 1\\
 & $\bullet$ point 1\\
 & $\bullet$ point 1\\
\textbf{Feb. 04} & \textbf{\textit{Finalize Sensor Simulations?}}\\
 & $\bullet$ point 1\\
 & $\bullet$ point 1\\
 & $\bullet$ point 1\\
\textbf{Feb. 11} &\textbf{Preliminary Sensor Selection Deliverable}\\
 & $\bullet$ point 1\\
 & $\bullet$ point 1\\
 & $\bullet$ point 1\\
\textbf{Feb. 18} &\textbf{\textit{Research Applicable Actuators?}}\\
 & $\bullet$ point 1\\
  & $\bullet$ point 1\\
   & $\bullet$ point 1\\
\textbf{Feb. 25} & \textbf{\textit{Do Design?}}\\
 & $\bullet$ point 1\\
  & $\bullet$ point 1\\
   & $\bullet$ point 1\\
\textbf{Mar. 04} & \textbf{\textit{Finalize Actuator Simulations?}}\\
 & $\bullet$ point 1\\
  & $\bullet$ point 1\\
   & $\bullet$ point 1\\
\textbf{Mar. 11} &\textbf{Preliminary Actuator Selection Deliverable}\\
 & $\bullet$ point 1\\
  & $\bullet$ point 1\\
   & $\bullet$ point 1\\
\textbf{Mar. 18} & \textbf{\textit{Evaluate Preliminary Sensor and Actuators?}}\\
 & $\bullet$ point 1\\
  & $\bullet$ point 1\\
   & $\bullet$ point 1\\
\textbf{Mar. 25} & \textbf{\textit{Obtain feedback and iterate?}}\\
 & $\bullet$ point 1\\
  & $\bullet$ point 1\\
   & $\bullet$ point 1\\
\textbf{Apr. 01} & \textbf{Finalize Final Simulations and Report}\\
 & $\bullet$ point 1\\
  & $\bullet$ point 1\\
   & $\bullet$ point 1\\


\end{tabular}
}



\end{document}
