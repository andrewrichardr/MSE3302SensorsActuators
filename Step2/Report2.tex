\documentclass[12pt]{article}
\usepackage[letterpaper, margin=0.75in]{geometry}
\usepackage{blindtext}
\usepackage[utf8]{inputenc}
\usepackage{graphicx} 
\graphicspath{ {./} }
\usepackage{float} 
\usepackage{karnaugh-map}
\usepackage{rotating}
\usepackage{pdflscape}
\usepackage{caption}
\usepackage{booktabs}
\usepackage{subcaption}
\usepackage{placeins}
\usepackage{gensymb}

\newcommand{\foo}{\hspace{-2.3pt}$\bullet$ \hspace{5pt}}
\newcommand{\ts}{\textsuperscript}

\begin{document}
\title{\textbf{Tesseract Project Revisit\\Preliminary Sensor Selection\\ MSE 3302 B}}
\author{
  Alan Harris\\
  250901911
  \and
  Robert Potra\\
  250914807
  \and
  Andrew Randell\\
  250911270
  \and
  Kevin Wang\\
  250908180
}
\date{\today}
\maketitle

\tableofcontents
\thispagestyle{empty}

\pagebreak
\setcounter{page}{1}

\section{Sensor Specifications}
\subsection{Local Position relative to Perimeter}
\subsubsection{Concept Generation}

\begin{enumerate}
\item \textbf{LIDAR} - This method involves using pulsing lasers to scan the perimeter, and using response times to map out the surrounding area.

\item \textbf{Ultrasonic} - This sensor placed on the outside of the vehicle uses sound to detect nearby objects, and avoid obstacle collisions.

\item \textbf{Sonar/Radar} - This method is similar to ultrasonic, but on a much larger scale. It uses the same principles, but will allow the robot to gather a larger image of the perimeter.
\end{enumerate}


\subsection{Local Position relative to Tesseract}


\subsection{Local Position relative to Pyramid(s)}
\subsection{Global Position}
From a macro perspective, the vehicle will need to know an approximate location of where it is on a map, as well as the approximate locations of pyramid(s) and tesseracts. 

\paragraph{GNSS} (Global Navigation Satellite System) is the global positioning technology that is applicable. GNSS is a generic world-wide term used to describe satellite navigation systems. The specific technology used is dependant on the region where it is being used. Regional tecnologies include GPS, GLONASS, Beidou, and Galileo.  This technology provides \textit{approximate} latitude, longitude, and altitude metrics to their host device. 

\paragraph{Hardware Implementation} of GNSS on devices is very straight forward with GNSS Modules. These modules are integrated GNSS receivers which can easily be implemented onto devices. They generally require a power input and provide NMEA 0183 GNSS coordinates over a UART connection. A table below has been compiled of high ranking GNSS modules. The controller must support UART communication for compatibility with most GNSS modules.

\paragraph{Software Implementation} will require prior mapping of the power plant. The GNSS will provide approximate coordinates on this software map. This will be used in conjunction with local sensing techniques to make informed decisions about the locations of the autonomous system relative to the tesseract and pyramid(s).


\section{Controller}

\subsection{Navigational Map} The robot completes a survey of the area creating a software map in memory. All sensor readings are included in this survey. After the survey has been completed, the system will have a map of the area in which it operates. This map will continuously be updated as the system operates. The map can by used for autonomous navigation.

\section{Modeling}

\clearpage
\section{Project Timeline}

\begin{flushleft}
\textit{Week of:}

\begin{tabular}{l | l}


 
\textbf{Feb. 11} &\textbf{Preliminary Sensor Selection Deliverable: Step 2 Due Feb. 15\ts{th}}\\
 & $\bullet$ Prepare and finalize report for step 2 deliverable\\\\
 
\textbf{Feb. 18} &\textbf{Research Applicable Actuators}\\
 & $\bullet$ Restate and redefine actuator specifications\\
  & $\bullet$ Identify possible actuator options based on previous concepts\\\\
  
\textbf{Feb. 25} & \textbf{Continue Actuator Design}\\
 & $\bullet$ Concept generation using possible actuator options\\
  & $\bullet$ Refine logic connecting the sensor data to actuator actions\\\\
  
\textbf{Mar. 04} & \textbf{Finalize Actuator Simulations}\\
 & $\bullet$ Perform actuator simulations using Simulink\\
 & $\bullet$ Begin preparing report for step 3 deliverable\\
 & $\bullet$ Perform concept selection using actuator simulation results and analysis\\\\
 
\textbf{Mar. 11} &\textbf{Preliminary Actuator Selection Deliverable: Step 3 Due Mar. 15\ts{th}}\\
 & $\bullet$ Continue preparing and finalize report step 3 deliverable\\\\
 
\textbf{Mar. 18} & \textbf{Evaluate Sensor and Actuators}\\
 & $\bullet$ Evaluate the proposed system of sensors and actuators\\
  & $\bullet$ Create a kinematic system model and perform analysis using Solidworks\\\\
  
\textbf{Mar. 25} & \textbf{Obtain Feedback and Iterate}\\
  & $\bullet$ Identify possible problems with the proposed system of sensors and actuators\\
  & $\bullet$ Refine analysis for the transducers, control device, kinematics, and power supply\\\\
  
\textbf{Apr. 01} & \textbf{Finalize Final Simulations and Report: Step 4 Due Apr. 5\ts{th}}\\
 & $\bullet$ Continue preparing final report for step 4 deliverable\\

\end{tabular}
\end{flushleft}


\end{document}